\documentclass{article}
\usepackage[margin=3cm]{geometry}
\usepackage[dvipsnames]{xcolor}
\usepackage[hidelinks]{hyperref}
\hypersetup{
    colorlinks=true,
    linkcolor=gray,
    filecolor=magenta,      
    urlcolor=RoyalBlue,
    pdftitle={Overleaf Example},
    pdfpagemode=FullScreen,
    }
\title{\bfseries\Huge Toghrul Karimov}
\author{\{email: toghs@mpi-sws.org, website: toghrul-karimov.github.io\}}
\date{}
\begin{document}
	\maketitle
		
	\section*{Employment}
	\begin{minipage}{0.3\textwidth}
		\hspace{0.1cm} \textbf{04/2025 - present}
	\end{minipage}
	\vspace*{0.25cm}
	\begin{minipage}{0.7\textwidth}
		Postdoctoral researcher working with Val\'erie Berth\'e 
		
		\vspace*{0.2cm}
		
		IRIF, CNRS, Paris and Max Planck Institute for Software Systems (MPI-SWS), Saarbr\"ucken, Germany
		
		\vspace*{0.2cm}
			
		Funded by the ERC Synergy Grant ``DnyAMiCs''
		\vspace{0.3cm}
	\end{minipage}
	\vspace{0.85cm}
	\begin{minipage}{0.3\textwidth}
		\hspace{0.1cm} \textbf{03/2024 - 03/2025}
	\end{minipage}
	\begin{minipage}{0.7\textwidth}
		Postdoctoral researcher working with Jo\"el Ouaknine
		
		\vspace*{0.2cm}
		MPI-SWS, Saarbr\"ucken, Germany
	\end{minipage}
	
	\section*{Education}
		\begin{minipage}{0.3\textwidth}
			\hspace{0.5cm} \textbf{2019-2024}
		\end{minipage}
		\vspace*{0.25cm}
		\begin{minipage}{0.7\textwidth}
			PhD student at Saarland University and the MPI-SWS, Germany
			
			\vspace*{0.2cm}
			Supervisor: Jo\"el Ouaknine 
			
			\vspace*{0.2cm}
			Thesis: Algorithmic verification of linear dynamical systems
			
			\vspace*{0.2cm}
			Received the grade \emph{summa cum laude}, nominated for various dissertation awards (outcomes pending)
                \vspace{0.3cm}
		\end{minipage}
		\vspace{0.85cm}
		\begin{minipage}{0.3\textwidth}
		\hspace{0.5cm} \textbf{2015-2019}
	\end{minipage}
	\begin{minipage}{0.7\textwidth}
		MCompSci Computer Science, University of Oxford, United Kingdom
		
		\vspace*{0.2cm}
		First Class Honours
	\end{minipage}
	\section*{Research areas}
	\begin{itemize}
		\item I am a theoretical computer scientist working on decision problems that lie at the intersection of number theory, dynamical systems, logic, and automata theory.
		\item My PhD thesis was about decision problems that arise in verification of linear dynamical systems.
	\end{itemize}
	
	\section*{Scholarships and awards}
	\begin{enumerate}
		\item CPEC (see \url{www.perspicuous-computing.science}) mini-project award for a two-week research visit to Oxford University; Deutsche Forschungsgemeinschaft grant 389792660.
		\item Keble College Scholarship, 2016-2019. Awarded for excellent performance in exams at the end of each year.
		\item The Scholarship of the Ministry of Education of Azerbaijan covering the full costs of my study at the University of Oxford, 2015-2019.
	\end{enumerate}
	
	\section*{Teaching}
	\noindent\begin{minipage}{0.3\textwidth}
		\hspace{0.5cm} \textbf{Summer 2020}
	\end{minipage}
	\begin{minipage}{0.7\textwidth}
		``Automata and sequences'', teaching assistant 
		
		University of Saarland
	\end{minipage}
	
	\vspace{0.3cm}
	
	\noindent\begin{minipage}{0.3\textwidth}
		\hspace{0.5cm} \textbf{Winter 2022}
	\end{minipage}
	\begin{minipage}{0.7\textwidth}
		``Topics in algorithmic dynamical systems theory'', teaching assistant 
		
		University of Saarland
	\end{minipage}

      

    
	\section*{Talks and presentations}
	\begin{enumerate}
		%\item Computational problems in dynamical systems theory. Azerbaijan University of Architecture and Construction, Baku, Azerbaijan.
            \item From word combinatorics to automatic structures. \emph{Workshop on Recent Developments in Arithmetic Theories and Applications}, Kolkata, India, 2025.
            \item On the decidability of Presburger arithmetic expanded with powers. \emph{SODA 2025, New Orleans, United States.}
		\item Ode to o-minimality. \emph{Symbolic Dynamics and Arithmetic Expansions} workshop in Roscoff, France, and \emph{Stellenbosch University logic seminar}, South Africa, 2024.
		\item The power of Positivity. \emph{LICS 2023}, Boston, United States.
		\item The model-checking problem for linear dynamical systems. \emph{Bellairs 2023} workshop in Barbados.
		\item The pseudo-reachability problem for diagonalisable affine dynamical systems. \emph{MFCS 2022}, Vienna, Austria, and \emph{RP 2022}, Saarbr\"ucken, Germany.
		\item The pseudo-Skolem problem is decidable. \emph{MFCS 2021}, Tallinn, Estonia.
		\item Invariants and impossibility: from geometric constructions to solving polynomial equations. \emph{Monsoon Math 2021}, an online camp for Indian students.
		\item Deciding $\omega$-regular properties on linear recurrence sequences. \emph{POPL 2021}.
		\item On verification of linear dynamical systems. \emph{Lighthning Talk at MPI-SWS, 2020, Saarbr\"ucken, Germany.}
		\item  On LTL model-checking for low-dimensional discrete
		linear dynamical systems. \emph{MFCS 2020}.
	\end{enumerate}
\end{document}